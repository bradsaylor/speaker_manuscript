\documentclass[11pt]{book}
\usepackage{mystyle}

\begin{document}
\section*{Introduction}

\section*{Transduction in the dynamic loudspeaker}

The dynamic loudspeaker system is often refered to as a 'transducer,'
and this is true as it transforms electrical power into acoustic
power.  However, there are infact two stages of transduction that
occur within the system.  These are electrical to mechanical power,
and mechanical to acoustic power.

Each stage is governed by physical laws that describe the key
parameters and relationships involved.  To achieve an accurate
equivalent circuit of the system one must have the knowledge of these
laws and how each parameter relates to its curcuit equivaletnt.

\section*{Equations of transduction}

When a current is presented to the speaker voice coil the electrons in
motion interact with the stationary field in the magnetic gap.  A
force is produced which is perpendicular to both the current flow and
the stationary magnetic field.  The magnetic filed is in the radial
direction while the current is always tangent to the voice coil
circumference.  The resulting force is always axial with
respect to the voice coil, pushing either inward or outward.  The
relationship is described in the following equation

\begin{equation*}
  \vec{F} = \vec{B} \ell \times \vec{i}
\end{equation*}

Where $\vec{B}$ and  $\ell$ are the magnetic gap field and length of wire
in the magnetic gap respectively.  The multiplicaion here is a vector
cross product, which has the magnitude

\begin{equation*}
  |\vec{F}| = |B\ell i sin(\theta)| = B\ell i
\end{equation*}

where $\theta$ is the angle between $\vec{B}$ and $\vec{i}$.  As
stated these two vectors are perpendicular so the $|sin(\theta)|$ term
must be unity.  The direction of $\vec{F}$ is, therefore, either
inwards or outwards and determined by application of the right hand
rule.

Transduction between the mechanical and acoustical domains is more
straightforward.  An acoustic pressure is generated due
to the mechanical force derived above acting over the area of the
speaker cone or

\begin{equation*}
  \vec{P} = \vec{F} * S_D
\end{equation*}

where $S_D$ is the effective cone surface area.  This is essentially
the area calculated by considering the cone to be a circle with a diameter
stretching from one surround peak across to the other.

With these laws of transduction in hand we will be able to translate
between electrical, mechanical, and acoustic domains in the eventual
analogous circuit
\section*{Method of construction}

We now consider the method by which an analogous circuit may be
constructed.  A logical first question would be whether or not there
is a unique circuit representation for the mechanical analogy.  As it
turns out there are two distinct possibilities.

\subsection*{Dual circuits}

Consider the development of the L-C-R cicuit response as given
previously.  When constructed as a series circuit it was clear that
current obeyed the same underlying differential equation as required
by displacement in the mechanical system and a formal parallel was
inferred.

In contrast, let us consider the parallel L-C-R circuit

\begin{figure}[h!]
  \centering
    \begin{circuitikz}[scale=1.3]
      \draw (0,0)
      to[R=$R$](0,2)
      to[short] (2,2)
      to[L=$L$] (2,0)
      to[short] (0,0);
      \draw (2,2)
      to[short] (4,2)
      to[C=$C$] (4,0)
      to[short] (2,0);
    \end{circuitikz}
   \caption{Sourceless parallel L-C-R circuit}
 \end{figure}

 We determine the voltage response of this circuit by considering the
 current through each leg.

 \begin{equation*}
   C \frac{de}{dt} + \frac{1}{R}e + \frac{1}{L} \int e \,dt = 0
 \end{equation*}

 taking the time derivative

 \begin{equation*}
   C \frac{d^2e}{dt^2} + \frac{1}{R} \frac{de}{dt} + \frac{1}{L}e = 0
 \end{equation*}

 and again we have found the equation for harmonic oscillation, albeit
 with some swapping and inverting of constant values.  With enough
 care in mapping mass, damping, and spring constants we may just as
 correctly use the parallel circuit to represent mechanical
 displacement in terms of voltage.  In general the mapping now becomes

 \begin{align*}
     \Lambda_2 &\Rightarrow \text{Mass} \Rightarrow \text{Capacitance}\\
     \Lambda_1 &\Rightarrow \text{Damping Constant} \Rightarrow \text{1/Resistance}\\
     \Lambda_0 &\Rightarrow \text{Spring Constant} \Rightarrow \text{1/Inductance}\\
 \end{align*}

 Circuits with this relationship are said to be duals of one another.
 As we have seen this transformation takes the current behavior of one
 circuit and transforms it to an identical voltage behavior in
 another.  As a last point we also note that, should a driving voltage or
 current source have been included, these would also swapped one for
 the other.  A voltage source becomes a current source of the same
 magnitude and vice versa.

 Back to our analogy construction we see we have a choice to make.
 Should the equivalent circuit be constructed as a parallel or series
 L-C-R type?  The decision is arbitrary and often a matter of
 convenience though care is needed to keep track of the underlying
 parameter mappings.
\subsection*{Choice of variables}

In our analagous circuit we are most concerned with how electrical
power at the input is translated to acoustical power at the output.
In fact, the electrical circuit gives us power directly by considering
the product

\begin{equation*}
  P_E = ei
\end{equation*}

Ideally our mechanical circuit should map equivalent parameters whose
product also indicates mechanical power.  This will become important
when the domains are linked via the equations of transduction above.
At present we have been concerned only with mechanical displacement.
If displacement is mapped either to current (or voltage) in an
equivalent circuit we may ask what mechanical equivalent is
represented by voltage (or current).  Further, would the product of
displacement and this variable yield mechanical power?

This line of reasoning suggests we try another approach.  Let us
consider the mass-spring system equation in terms of velcity instead
of displacement

\begin{equation*}
  m\frac{dv}{dt} + bv + k\int v \,dt = 0
\end{equation*}

Already we can see a closer similarity to the L-C-R circuit
equivalent.  Taking the time derivative reveals the same equation of
harmonic oscillation

\begin{equation*}
  m\frac{d^2v}{dt^2} + b\frac{dv}{dt} + kv = 0
\end{equation*}

As a result we see that mechanical velocity may be used as the analog
to electrical current or voltage in our chosen analogy.  Taking the
dual of this equation transforms the dependant variable into its power
product partner

\begin{equation*}
  \frac{1}{k}\frac{d^2\Delta}{dt^2} + \frac{1}{b}\frac{d\Delta}{dt} + \frac{1}{m}\Delta = 0
\end{equation*}

For the sake of clarity we integrate both sides

\begin{equation*}
  \frac{1}{k}\frac{d\Delta}{dt} + \frac{1}{b}\Delta + \frac{1}{m}\int \Delta \,dt = 0
\end{equation*}

Now each term here must be a velocity for the dual
relationship to hold.  Using this information we may now determine the
variable mapped to $\Delta$.  Take, for instance, the integral term

\begin{equation*}
  \frac{1}{m}\int \Delta \,dt = v
\end{equation*}

so that $\int \Delta \,dt$ must have units of mass times velocity or
momentum.  Straight away this imples that

\begin{equation*}
  \Delta \quad \Rightarrow \quad Force
\end{equation*}

and as required mechanical power is indeed given by

\begin{equation*}
  P_M = Fv
\end{equation*}

We will proceed in using velocity and force to represent current and
voltage in our equivalent circuit.

\subsection*{Impedance and mobility analogies}


The two equivalent circuit analogies above have specific names.  These
are the impedance type analogy and the mobility type analogy.  To
distinguish between the two we look at the fundamental differential
equation for the system.

If we were to choose to map velocity to current and force to voltage
this would correspond to the series L-C-R configuration.  This stems
from the fact that a common velcity must be shared by both the spring
and mass just as a common current is shared among the series cicuit
elements.  The system equation for this scenario exhibits the damping
term $bv$ so that $b$ is interpereted as a mechanical impedance.
Hence this is the impedance analogy

Conversely if velocity is mapped to voltage the restulting circuit is
the parallel L-C-R configuration.  Again the velcity is common to all
mechanical elements in the same way that the voltage is common across
all three circuit elements.  The system equation here, as shown, has a
damping term of $\frac{1}{b}F$ so that $b$ is interpreted as an
inverse impedance, also called mobility.  This is the mobility analogy.




\subsection*{Transformation between domains}

The dynamic loudspeaker system contains three physical domains namely
electrical, mechanical, and acoustical.  Transduction occurs when
moving from one domain to another and is governed by the equations
already given.  If each domain is to have an equivalent circuit then
we must determine how to join these three circuits into one model.  At
the transition point power from one domain is converted
into an equal power in the next.  In traditional circuit theory the
coupling of power from one circuit to another may be accomplished by
the transformer element.

\begin{center}
\begin{circuitikz} [american]
  \draw
  (0,0) node[transformer core] (T) {}
  (T.base) node{1:n}
  (T.B2) to[short] (2, -2.1)
  (T.B1) to[short,i_>=$i_2$] (2, 0)
  (T.A2) to[short] (-2,-2.1)
  (T.A1) to[short,i<=$i_1$] (-2,0)
  ;

  \draw
  (-.5,0) to[open,v=$e_1$] (-.5,-2)
  (1.2,0) to[open,v=$e_2$] (1.2,-2)
  ;
  
\end{circuitikz}
\end{center}

For a windings ratio of $n$ the voltage and current ratios are as
follows

\begin{align*}
  v_2 &= nv_1\\
  i_2 &= \frac{i_1}{n}
\end{align*}

and by the electical power equation $P_E$ is conserved in the
translation.  For our analgous circuit the windings ratio is replaced
by the current or voltage conversion factor.  For, example mapping
mechanical force to current and mechanical velocity to voltage and
using the equation of transduction between electrical and mechanical
domains gives

\begin{center}
\begin{circuitikz} [american]
  \draw
  (0,0) node[transformer core] (T) {}
  (T.base) node{$B\ell$:1}
  (T.B2) to[short] (2, -2.1)
  (T.B1) to[short,i_>=$F$] (2, 0)
  (T.A2) to[short] (-2,-2.1)
  (T.A1) to[short,i<=$i$] (-2,0)
  ;

  \draw
  (-.5,0) to[open,v=$e$] (-.5,-2)
  (1.2,0) to[open,v=$v$] (1.2,-2)
  ;
  
\end{circuitikz}
\end{center}






\subsection*{The driver electrical circuit}


It is finally time to put pen to paper to form our circuit model.  We
begin with the driver electrical domain.  Physically this consists of
the voice coil only.  Electrically this breaks down into two
elements.  As a length of copper wire the voice coil has resistance in
ohms determined by its length and diameter.  This is commonly known as
$R_e$.  Because this wire is wound in loops
with a core of air and steel (pole piece) it also behaves as an
inductor.  Putting these together with an input voltage source, an
amplifier for instance, we obtain



  \begin{center}
  \begin{circuitikz}[scale=1.3, american]
      \draw
      (3.5,1.6) node[transformer core] (T) {}
      ;
      \draw (0,0)
      to[sV, l=$e$](0, 1.6)
      to[R=$R_e$] (1.5, 1.6)
      to[L=$L_e$] (T.A1)
      ;
      \draw (0,0)
      to[short] (T.A2)
      ;
 \end{circuitikz}
\end{center}

 Here the transformer has been added in anticipation of adding the
 mechanical domain
 
\subsection*{Adding the mechanical domain}
Since our equation of transduction between electrical and mechanical
domains relates current to mechanical force we choose the mobility
analogy.  The quantity $B\ell$ in the electrical domain must map to unity
in the mechancial domain giving

  \begin{center}
  \begin{circuitikz}[scale=1.3, american]
      \draw
      (3.5,1.6) node[transformer core] (T) {}
      (T.base) node{1:$B\ell$}
      ;
      \draw (0,0)
      to[sV, l=$e$] (0, 1.6)
      to[R=$R_e$] (1.5, 1.6)
      to[L=$L_e$] (T.A1)
      ;
      \draw (0,0)
      to[short, i_<=$i$] (T.A2)
      ;
      \draw (T.B1)
      to [short] (4.75, 1.6)
      to [R=$\frac{1}{R_{MS}}$] (4.75, 0)
      ;
      \draw (4.5, 1.6)
      to [open, v=$v$] (4.5, 0)
      ;
      \draw (4.75, 1.6)
      to [short, i^>=$F$] (6, 1.6)
      to [L=$L_C$] (6, 0)
      ;
      \draw (6, 1.6)
      to [short] (7.25, 1.6)
      to [C=$C_M$] (7.25, 0)
      ;
      \draw (7.25, 1.6)
      to [short] (8, 1.6)
      ;
      \draw
      (8.75, 1.6) node[transformer core] (T2) {}
      ;
      \draw (T2.A2)
      to [short] (T.B2)
      ;
 \end{circuitikz}
\end{center}

As discussed mass corresponds to capacitance in the model and
inductance corresponds to compliance.

\subsection*{Adding the acoustic domain}

In the acoustic domain we are dealing with pressure and volume
velocity as the power product pair.  It is most intuitive to map
pressure to voltage and volume velocity to current.  This is the
impedance analogy and takes the form



  \begin{center}
  \begin{circuitikz}[scale=1.3, american]
      \draw
      (3.5,1.6) node[transformer core] (T) {}
      (T.base) node{1:$B\ell$}
      ;
      \draw (0,0)
      to[sV, l=$e$] (0, 1.6)
      to[R=$R_e$] (1.5, 1.6)
      to[L=$L_e$] (T.A1)
      ;
      \draw (0,0)
      to[short, i_<=$i$] (T.A2)
      ;
      \draw (T.B1)
      to [short] (4.75, 1.6)
      to [R=$R_{MS}$] (4.75, 0)
      ;
      \draw (4.5, 1.6)
      to [open, v=$v$] (4.5, 0)
      ;
      \draw (4.75, 1.6)
      to [short, i^>=$F$] (6, 1.6)
      to [L=$L_C$] (6, 0)
      ;
      \draw (6, 1.6)
      to [short] (7.25, 1.6)
      to [C=$C_M$] (7.25, 0)
      ;
      \draw (7.25, 1.6)
      to [short] (8, 1.6)
      ;
      \draw
      (8.75, 1.6) node[transformer core] (T2) {}
      (T2.base) node{$S_D$:1}
      ;
      \draw (T2.A2)
      to [short] (T.B2)
      ;
      \draw (T2.B1)
      to [short] (10, 1.6)
      to [generic=$\frac{1}{Z_f}$] (10, 0)
      ;
      \draw (9.7, 1.6)
      to [open, v=$U$] (9.7, 0)
      ;
      \draw(10, 1.6)
      to [short] (11, 1.6)
      to [generic=$\frac{1}{Z_b}$] (11, 0)
      to [short, i^>=$p$] (T2.B2)
      ;
 \end{circuitikz}
\end{center}

Here $Z_f$ and $Z_b$ correspond to the acoustic radiation impedances
of the front and back of the driver respectively.  These are,
generally, complex but will be left in these simplified terms for the
time being.



\subsection*{Completing the analogy}
We have essentially arrived at our goal of a circuit based system
model.  At this point  we may transform this cicuit into a form more
suitable for analysis.  Our transformer notation shows the underlying
relationships between domains.  For the sake
of analysis, however, it would be better if they were not required.
Each circuit element represents an impedance relating current to
voltage.  The transformer equations developed above tell us how to
transform impedances thereby obviating the use of the transformer
element.  The impedance relationship becomes

\begin{equation*}
  Z=e/i \quad \Rightarrow \quad Z_T=(n*e)/\left(\frac{i}{n}\right) = Zn^2
\end{equation*}


Moving back to our circuit we wish to convert the entire model
into the acoustic domain.  This may be accomplished in two steps.
First we combine the mechanical and electrical domains by dividing the
electrical domain impedances by the traformer conversion factor
$(B\ell)^2$ giving


  \begin{center}
  \begin{circuitikz}[scale=1.3, american]
      \draw (0,0)
      to[sV, l=$\frac{e}{(B\ell)}$] (0, 1.6)
      to[R=$\frac{R_e}{(B\ell)^2}$] (1.5, 1.6)
      to[L=$\frac{L_e}{(B\ell)^2}$] (3, 1.6)
      ;
      \draw (1.5, 1.6)
      to[open, v=$v$] (1.5, 0)
      ;
      \draw (3, 1.6)
      to [R=$\frac{1}{R_{MS}}$] (3, 0)
      ;
      \draw (3, 1.6)
      to [short, i^>=$F$] (4.5, 1.6)
      to [L=$L_C$] (4.5, 0)
      ;
      \draw (4.5, 1.6)
      to [short] (6, 1.6)
      to [C=$C_M$] (6, 0)
      ;
      \draw (6, 1.6)
      to [short] (7, 1.6)
      ;
      \draw
      (7.75, 1.6) node[transformer core] (T2) {}
      (T2.base) node{$S_D$:1}
      ;
      \draw (T2.A2)
      to [short] (0, 0)
      ;
      \draw (T2.B1)
      to [short] (9, 1.6)
      to [generic=$\frac{1}{Z_f}$] (9, 0)
      ;
      \draw (8.75, 1.6)
      to[open, v=$U$] (8.75, 0)
      ;
      \draw(9,1.6)
      to [short] (10, 1.6)
      to [generic=$\frac{1}{Z_b}$] (10, 0)
      to [short, i^>=$p$] (T2.B2)
      ;
 \end{circuitikz}
\end{center}


The electro-mechanical domain may then be combined with the acoustical
domain using the transformation factor $S_D{}^2$ giving

  \begin{center}
  \begin{circuitikz}[scale=1.3, american]
      \draw (0,0)
      to[sV, l=$\frac{eS_D}{(B\ell)}$] (0, 1.6)
      to[R=$\frac{R_eS_D{}^2}{(B\ell)^2}$] (1.5, 1.6)
      to[L=$\frac{L_eS_D{}^2}{(B\ell)^2}$] (3, 1.6)
      ;
      \draw (1.5, 1.6)
      to[open, v=$U$] (1.5, 0)
      ;
      \draw (3, 1.6)
      to [R=$\frac{S_D{}^2}{R_{MS}}$] (3, 0)
      ;
      \draw (3, 1.6)
      to [short] (4.75, 1.6)
      to [L=$L_CS_D{}^2$] (4.75, 0)
      ;
      \draw (4.75, 1.6)
      to [short, i^>=$p$] (6.5, 1.6)
      to [C=$\frac{C_M}{S_D{}^2}$] (6.5, 0)
      ;
      \draw (6.5, 1.6)
      to [short] (8.5, 1.6)
      ;
      \draw (8.5, 1.6)
      to [generic=$\frac{1}{Z_f}$] (8.5, 0)
      ;
      \draw (8.5 ,1.6)
      to [short] (9.5, 1.6)
      to [generic=$\frac{1}{Z_b}$] (9.5, 0)
      to [short] (0,0)
      ;
            
 \end{circuitikz}
\end{center}

Note that the mechanical mass-capacitance takes the factor
$\frac{1}{S_D{}^2}$ because the impedance of the capacitor is
proportional to the inverse of capacitance, namely
$\frac{1}{j\omega C}$

One more transformation is now considered.  In present form we have
mapped volume velocity to voltage and pressure to current.  It will be
preferrable in the analysis to come to explore the opposite mapping.
We need to map volume velocity to current and pressure to voltage.
This will give the more intuitive impedance analogy of the entire
circuit.

This may be accomplished in two steps.  First the driving voltage,
voice coil resistance, and inductance are taken into the mobility
analogy.  After this is complete the entire circuit will be of the
mobility form.  Then we may take the dual of the entire circuit
transforming from mobility to impedance analogies and obtaining the
desired voltage and current mapping.

By Norton's Theorm of equivalence the series voice coil resistance and
inductance may be represented as an equivalent parallel circuit. The
voltage source is replaced by a parallel current source whose value is
the same as the short circuit current of the orignal series
configuration.  Calculating the short circuit current in series gives



  \begin{center}
  \begin{circuitikz}[scale=1.3, american]
    \draw (0, 0)
    to[sV, l=$\frac{eS_D}{(B\ell)}$] (0, 1.6)
    to[R=$\frac{R_eS_D{}^2}{(B\ell)^2}$] (1.5, 1.6)
    to[L=$\frac{L_eS_D{}^2}{(B\ell)^2}$, -o] (3, 1.6)
    ;
    \draw [dashed] (3, 1.6)
    to[short,i=$i_{SC}\Rightarrow\frac{(B\ell)}{S_D}\left(\frac{e}{R_e
        + j\omega L_e}\right)$]  (3, 0)
    ;
    \draw (3, 0)
    to[short, o-] (0, 0)
    ;
 \end{circuitikz}
\end{center}

Replacing this section with its parallel equivalent gives

  \begin{center}
  \begin{circuitikz}[scale=1.3, american]
      \draw (0,0)
      to[I, l=$\frac{(B\ell)}{S_D}\left(\frac{e}{R_e + j\omega
          L_e}\right)$] (0, 3.2)
      to[short] (3.5, 3.2)
      ;
      \draw (1.5, 3.2)
      to[R=$\frac{R_eS_D{}^2}{(B\ell)^2}$] (1.5, 1.6)
      to[L=$\frac{L_eS_D{}^2}{(B\ell)^2}$] (1.5, 0)
      ;
      \draw (1, 3.2)
      to[open, v=$U$] (1, 0)
      ;
      \draw (3.5, 3.2)
      to [R=$\frac{S_D{}^2}{R_{MS}}$] (3.5, 0)
      ;ppp
      \draw (3.5, 3.2)
      to [short] (4.75, 3.2)
      to [L=$L_CS_D{}^2$] (4.75, 0)
      ;
      \draw (4.75, 3.2)
      to [short, i^>=$p$] (6.5, 3.2)
      to [C=$\frac{C_M}{S_D{}^2}$] (6.5, 0)
      ;
      \draw (6.5, 3.2)
      to [short] (7.75, 3.2)
      ;
      \draw (7.75, 3.2)
      to [generic=$\frac{1}{Z_f}$] (7.75, 0)
      ;
      \draw (7.75, 3.2)
      to [short] (8.75, 3.2)
      to [generic=$\frac{1}{Z_b}$] (8.75, 0)
      to [short] (0,0)
      ;
            
 \end{circuitikz}
\end{center}
            
We are now ready for the final step.  Taking the dual of our circuit
gives the desired relationships, voltage to pressure and current to
volume velocity.

\begin{center}
  \begin{circuitikz} [american, scale=1.3]
  \draw (0, 0)
  to[sV, l=$\frac{(B\ell)}{S_D}\left(\frac{e}{R_e
      + j\omega L_e}\right)$] (0, 1.6)
  to[short] (0.75, 1.6)
  to[R, l_=$\frac{(B\ell)^2}{R_eS_D{}^2}$] (2.25, 1.6)
  ;
  \draw (0.75, 1.6)
  to[short] (0.75, 2.1)
  to[C=$\frac{L_eS_D{}^2}{(B\ell)^2}$] (2.25, 2.1)
  to[short] (2.25, 1.6)
  ;
  \draw (2.25, 1.6)
  to[R, l=$\frac{R_{MS}}{S_D{}^2}$] (3.5, 1.6)
  to[C=$L_CS_D{}^2$] (5, 1.6)
  to[L=$\frac{C_M}{S_D{}^2}$] (6.5, 1.6)
  to[generic, l=$Z_f$] (8, 1.6)
  to[generic, l=$Z_b$] (8, 0)
  to[short, i>^=$U$] (0, 0)
  ;
  \draw (0.85, 1.6)
  to[open, v=$p$] (0.85, 0)
  ;
  \end{circuitikz}
\end{center}

For a given front and back acoustic impedance we may now relate the
input electrical voltage $e$ to the generated acoustic volume velocity
and pressure.  This is the form we shall use for further analysis of
the loudspeaker system behavior.

\end {document}



%%% Local Variables:
%%% mode: latex
%%% TeX-master: t
%%% End:
